% !TEX encoding = UTF-8
% !TEX TS-program = pdflatex
% !TEX root = ../tesi.tex

%**************************************************************
% Sommario
%**************************************************************
\cleardoublepage
\phantomsection
\pdfbookmark{Sommario}{Sommario}
\begingroup
\let\clearpage\relax
\let\cleardoublepage\relax
\let\cleardoublepage\relax

\chapter*{Sommario}

Il presente documento descrive il tirocinio da me svolto presso l'azienda San Marco Group S.p.A. nella sede di Marcon (VE), nel periodo che va dal 03-08-2020 al 06-11-2020. L'esperienza di stage ha avuto una durata complessiva di 320 ore ed è stata supervisionata e coordinata sia dal mio tutor aziendale, Mauro Vecchiato, che dal mio relatore presso l'ateneo, prof. Tullio Vardanega.\\
Lo scopo principale di questo progetto di stage consisteva nello sviluppo di un sistema di gestione di offerte di fornitura, dedicate a beni materiali e servizi.\\
In particolare, una volta analizzato il processo aziendale ed essermi interfacciato con gli utenti coinvolti per identificare la soluzione progettuale più adatta, ho progettato un'interfaccia del gestionale utilizzata dagli utenti interni all'azienda e sviluppato un'applicazione \textit{web} utilizzabile dai fornitori per rispondere alle richieste di offerta.\\
Il presente documento è suddiviso in 4 capitoli:
\begin{itemize}
	\item \textbf{Capitolo 1}: presentazione del contesto aziendale, processi interni e propensione all'innovazione;
	\item \textbf{Capitolo 2}: presentazione dell'offerta di stage e motivazioni della scelta effettuata; 
	\item \textbf{Capitolo 3}: presentazione dettagliata del progetto, con approfondimento delle sue fasi, delle tecnologie utilizzate e delle soluzioni attuate per realizzarlo;
	\item \textbf{Capitolo 4}: resoconto del lavoro svolto con una valutazione personale sugli obiettivi raggiunti, le difficoltà incontrate e le conoscenze personali e professionali acquisite.
\end{itemize}
Per la redazione del documento sono state adottate le seguenti norme tipografiche:
\begin{itemize}
\item i termini in lingua diversa dall'italiano sono scritti in corsivo;
\item i termini del Glossario presenti nel testo sono marcati con una G maiuscola a
pedice alla loro prima occorrenza e scritti in corsivo.
\end{itemize}

\vfill
%
%\selectlanguage{english}
%\pdfbookmark{Abstract}{Abstract}
%\chapter*{Abstract}
%
%\selectlanguage{italian}

\endgroup			

\vfill

