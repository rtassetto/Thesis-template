% !TEX encoding = UTF-8
% !TEX TS-program = pdflatex
% !TEX root = ../tesi.tex

%**************************************************************
% Sommario
%**************************************************************
\cleardoublepage
\phantomsection
\pdfbookmark{Sommario}{Sommario}
\begingroup
\let\clearpage\relax
\let\cleardoublepage\relax
\let\cleardoublepage\relax

\chapter*{Sommario}

Il presente documento descrive il \textit{stage} da me svolto presso l'azienda San Marco Group S.p.A. nella sede di Marcon (VE), nel periodo che va dal 03-08-2020 al 06-11-2020. L'esperienza di \textit{stage} ha avuto una durata complessiva di 320 ore ed è stata supervisionata e coordinata sia dal mio tutor aziendale, Mauro Vecchiato, che dal mio relatore presso l'ateneo, prof. Tullio Vardanega.\\
Lo scopo principale di questo progetto di \textit{stage} consisteva nello sviluppo di un sistema di gestione di offerte di fornitura, dedicate a beni materiali e servizi.\\
In particolare, una volta analizzato il processo aziendale ed essermi interfacciato con gli utenti coinvolti per identificare la soluzione progettuale più adatta, ho progettato un'interfaccia del gestionale utilizzata dagli utenti interni all'azienda e ho sviluppato un'applicazione \textit{web} utilizzabile dai fornitori per rispondere alle richieste di offerta.\\
Ho suddiviso il presente documento in 4 capitoli:
\begin{itemize}
	\item \textbf{Capitolo 1}: presentazione del contesto aziendale, processi interni e propensione all'innovazione;
	\item \textbf{Capitolo 2}: presentazione dell'offerta di \textit{stage} e motivazioni della scelta effettuata; 
	\item \textbf{Capitolo 3}: presentazione dettagliata del progetto, con approfondimento delle sue fasi, delle tecnologie utilizzate e delle soluzioni attuate per realizzarlo;
	\item \textbf{Capitolo 4}: resoconto del lavoro svolto con una valutazione personale sugli obiettivi raggiunti, le difficoltà incontrate e le conoscenze personali e professionali acquisite.
\end{itemize}
Per la redazione del documento ho adottato le seguenti norme tipografiche:
\begin{itemize}
\item ho scritto in corsivo i termini in lingua diversa dall'italiano;
\item ho scritto in corsivo e marcato con una G maiuscola a pedice i termini del Glossario presenti nel testo alla loro prima occorrenza.
\end{itemize}

\vfill
%
%\selectlanguage{english}
%\pdfbookmark{Abstract}{Abstract}
%\chapter*{Abstract}
%
%\selectlanguage{italian}

\endgroup			

\vfill

