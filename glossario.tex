
%**************************************************************
% Acronimi
%**************************************************************
\renewcommand{\acronymname}{Acronimi e abbreviazioni}

\newacronym[description={\glslink{apig}{Application Program Interface}}]
    {api}{API}{Application Program Interface}

\newacronym[description={\glslink{umlg}{Unified Modeling Language}}]
    {uml}{UML}{Unified Modeling Language}
    
\newacronym[description={\glslink{big}{Business Intelligence}}]
    {bi}{BI}{Business Intelligence}
    
\newacronym[description={\glslink{ictg}{Information and Communications Technology}}]
	{ict}{ICT}{Information and Communications Technology}
	
\newacronym[description={\glslink{erpg}{Enterprise Resource Planning}}]
	{erp}{ERP}{Enterprise Resource Planning}
	
\newacronym[description={\glslink{vpng}{Virtual Private Network}}]
	{vpn}{VPN}{Virtual Private Network}	
	
	

%**************************************************************
% Glossario
%**************************************************************
\renewcommand{\glossaryname}{Glossario}

\newglossaryentry{apig}
{
    name=\glslink{api}{API},
    text=Application Program Interface,
    sort=api,
    description={in informatica con il termine \emph{Application Programming Interface API} (ing. interfaccia di programmazione di un'applicazione) si indica ogni insieme di procedure disponibili al programmatore, di solito raggruppate a formare un set di strumenti specifici per l'espletamento di un determinato compito all'interno di un certo programma. La finalità è ottenere un'astrazione, di solito tra l'hardware e il programmatore o tra software a basso e quello ad alto livello semplificando così il lavoro di programmazione}
}

\newglossaryentry{ictg}
{
	name=\glslink{ict}{ICT},
	text=Information and Communications Technology,
	sort=ict,
	description={le tecnologie dell'informazione e della comunicazione, acronimo ICT, sono l'insieme dei metodi e delle tecniche utilizzate nella trasmissione, ricezione ed elaborazione di dati e informazioni}
}

\newglossaryentry{big}
{
	name=\glslink{bi}{Business Intelligence},
	text=Business Intelligence,
	sort=bi,
	description={Con la locuzione \textit{Business Intelligence} (\textit{BI}) ci si può solitamente riferire a tre elementi, quali: un insieme di processi aziendali per raccogliere dati ed analizzare informazioni strategiche; la tecnologia utilizzata per realizzare questi processi; le informazioni ottenute come risultato di questi processi}
}

\newglossaryentry{erpg}
{
	name=\glslink{erp}{Enterprise Resource Planning},
	text=Enterprise Resource Planning,
	sort=erp,
	description={Letteralmente tradotto "pianificazione delle risorse d'impresa", è un \textit{software} di gestione che integra tutti i processi di \textit{business} rilevanti di un'azienda e tutte le funzioni aziendali, ad esempio vendite, acquisti, gestione magazzino, finanza, contabilità, ecc.. Integra quindi tutte le attività aziendali in un unico sistema, il quale risulta essere indispensabile per supportare il \textit{Management}. I dati vengono raccolti in maniera centralizzata nonostante provengano da molteplici parti dell'azienda}
}

\newglossaryentry{vpng}
{
	name=\glslink{vpn}{Virtual Private Network},
	text=Virtual Private Network,
	sort=vpn,
	description={Acronimo di \textit{Virtual Private Network}, è un servizio che garantisce \textit{privacy}, anonimato e sicurezza dei dati attraverso un canale di comunicazione riservato tra dispositivi che non necessariamente devono essere collegati alla stessa \textit{LAN}. In ambito prettamente aziendale, una \textit{VPN} può essere paragonata ad una estensione geografica della rete locale privata (\textit{LAN}) e che, quindi, permette di collegare tra loro, in maniera sicura, i siti della stessa azienda dislocati sul territorio}
}



