
%**************************************************************
% Acronimi
%**************************************************************
\renewcommand{\acronymname}{Acronimi e abbreviazioni}

\newacronym[description={\glslink{apig}{Application Program Interface}}]
    {api}{API}{Application Program Interface}

\newacronym[description={\glslink{umlg}{Unified Modeling Language}}]
    {uml}{UML}{Unified Modeling Language}
    
\newacronym[description={\glslink{bi}{Business Intelligence}}]
    {bi}{BI}{Business Intelligence}
    
\newacronym[description={\glslink{ict}{Information and Communications Technology}}]
	{ict}{ICT}{Information and Communications Technology}

%**************************************************************
% Glossario
%**************************************************************
\renewcommand{\glossaryname}{Glossario}

\newglossaryentry{apig}
{
    name=\glslink{api}{API},
    text=Application Program Interface,
    sort=api,
    description={in informatica con il termine \emph{Application Programming Interface API} (ing. interfaccia di programmazione di un'applicazione) si indica ogni insieme di procedure disponibili al programmatore, di solito raggruppate a formare un set di strumenti specifici per l'espletamento di un determinato compito all'interno di un certo programma. La finalità è ottenere un'astrazione, di solito tra l'hardware e il programmatore o tra software a basso e quello ad alto livello semplificando così il lavoro di programmazione}
}

\newglossaryentry{ictg}
{
	name=\glslink{ict}{ICT},
	text=Information and Communications Technology,
	sort=ict,
	description={le tecnologie dell'informazione e della comunicazione, acronimo ICT, sono l'insieme dei metodi e delle tecniche utilizzate nella trasmissione, ricezione ed elaborazione di dati e informazioni}
}

\newglossaryentry{big}
{
	name=\glslink{bi}{Business Intelligence},
	text=Business Intelligence,
	sort=bi,
	description={Con la locuzione \textit{Business Intelligence} (\textit{BI}) ci si può solitamente riferire a tre elementi, quali: un insieme di processi aziendali per raccogliere dati ed analizzare informazioni strategiche; la tecnologia utilizzata per realizzare questi processi; le informazioni ottenute come risultato di questi processi}
}
