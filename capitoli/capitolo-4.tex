% !TEX encoding = UTF-8
% !TEX TS-program = pdflatex
% !TEX root = ../tesi.tex


%**************************************************************
\chapter{Conclusioni}
\label{cap:conclusioni}
%**************************************************************

%\intro{Breve introduzione al capitolo}\\


\section{Soddisfacimento degli obiettivi}

Ho portato a termine tutti gli obiettivi definiti a inizio \textit{stage}, nonostante mi sia mancata la possibilità di sviluppare meglio l'obiettivo desiderabile.
Infatti, sebbene il suo soddisfacimento, non mi ritengo completamente appagato del risultato raggiunto.\\
Era mio desiderio approfondire maggiormente l'analisi dei dati con i soggetti interessati e capire in che modo possono essere usati per stimolare il cambiamento ed eliminare le inefficienze.\\
Questo è dovuto dalla mancanza di tempo data dal prolungarsi dell'attività di verifica dei \textit{web service}, che ha richiesto più tempo del previsto sia per la difficoltà personale che ho trovato nell'affrontare la tecnologia sconosciuta, sia per alcuni problemi esterni all'azienda che hanno portato ad un rilascio posticipato del certificato digitale \textit{SSL} per il \textit{server web} di \textit{Sage X3}.


%**************************************************************


\section{Conoscenze acquisite}

\subsection{Conoscenze professionali}

A livello professionale, questa esperienza mi ha dato la possibilità di acquisire dimestichezza con \textit{framework} e tecnologie a me ancora sconosciute, che non rientrano nel contesto universitario ma sono molto utilizzare nel mondo lavorativo.\\
Ritengo sia stata molto utile l'esperienza avuta nell'analisi di un problema e nella comprensione di un processo aziendale, attività che nascondono una certa difficoltà anche per figure che hanno una lunga esperienza lavorativa alle spalle.\\
Il lavoro individuale richiesto dal progetto mi ha permesso di raggiungere un grado di autonomia che finora non avevo. Infatti, se faccio un confronto con l'esperienza del progetto svolto durante il corso di Ingegneria del Software, nonostante le attività fossero divise tra i vari membri del gruppo, c'era sempre la possibilità di confronto con colleghi che avevano la stessa esperienza e lo stesso \textit{background} di conoscenze.\\
Questo mi ha fatto capire allo stesso tempo quanto sia importante il confronto, a partire dagli incontri con il \textit{tutor} aziendale per affrontare assieme problematiche su cui ero carente dovute alla mia poca esperienza, fino al confronto con gli utenti finali nonché utilizzatori del prodotto, in quanto scelte sbagliate che non soddisfassero i loro bisogni avrebbero portato ad un fallimento del progetto.

%**************************************************************


\subsection{Conoscenze personali}

Personalmente questa esperienza è stata utile per migliorare il mio modo di rapportarmi con le persone, avendo a che fare con figure di diversa età e con molta esperienza alle spalle.\\
In questo modo ho potuto osservare e cercare di capire i modi di pensare di chi aveva più esperienza di me, cercando di attuarli nelle mie scelte e nel mio modo di pensare.\\
Inoltre ritengo di essere riuscito a sviluppare una discreta capacità di \textit{problem solving} considerando le varie scelte fatte durante il progetto, a partire dal cambio di progetto che ho dovuto attuare non appena terminata l'analisi e che mi ha costretto a rivedere interamente la pianificazione.\\
Oltre a questo, ritengo sia migliorata la mia capacità di gestire più attività parallelamente, dovendo passare dalle ore dedicate allo sviluppo del progetto, alle ore dedicate alle mansioni che richiedevano il mio ruolo.\\

%**************************************************************


\section{Valutazione personale}

Per quanto concerne la mia personale opinione su questa esperienza, posso dire di essere sufficientemente soddisfatto del risultato.\\
Mi sono reso conto della distanza tra il mondo accademico e quello lavorativo, trovandomi davanti all'utilizzo di tecnologie e strumenti che non vengono trattate durante il corso di studi.\\
Questa mancanza però è sopperita dalle nozioni teoriche ricevute e dalla forma mentis che l'università intende trasmettere.\\
Infatti, le conoscenze apprese durante il corso di studi, mi sono state sufficienti per riuscire a comprendere in autonomia l'utilizzo di \textit{framework} avanzati e tecnologie non ancora esplorate.\\
La parte che ho trovato più complessa è stata il rapporto con altre figure professionali. Sia per una mia mancanza personale dovuta al mio carattere introverso, quindi poco affine al rapporto con persone estranee, sia per la difficoltà di relazione con figure non tecniche e con esperienza lavorativa alle spalle.\\
È stato arduo cercare di capire le necessità di alcune attività molto lontane dall'ambiente universitario finora affrontato.\\
Tutto ciò mi ha motivato e mi ha permesso di conoscere e comprendere più a fondo quello che stavo facendo, dovendo trasmettere le mie idee a figure che operavano in un contesto diverso da quello informatico.\\
Un mio pensiero, datomi da questa esperienza, riguarda il ruolo dell'informatica nel mondo del lavoro.\\
Negli ultimi 30 anni la presenza dell'informatica nella vita di tutti i giorni e soprattutto il suo utilizzo all'interno dei processi aziendali è diventato fondamentale e il carattere di servizio e supporto si è delineato come cifra distintiva.\\
Questa caratteristica non è sicuramente negativa, perché anche se tutte le attività supportate dall'informatica potrebbero esistere indipendentemente da essa, nella società digitale attuale lo scarto si fa sempre più sottile e l'inscindibilità delle attività da un supporto digitale risulta irreale.\\
È per questo motivo che qualsiasi azienda che si occupi esclusivamente di informatica lo fa nella misura in cui essa possa offrire servizi informatici ad aziende che hanno come obiettivo la produzione di beni o servizi di altro genere; è perciò comune che, ad esempio, un'azienda di consulenza informatica crei servizi per aziende che producono, come nel mio caso, pitture e vernici.\\
Questo processo di supporto all'azienda potrebbe dunque essere tranquillamente dato in gestione a imprese esterne, ma il motivo per cui un'azienda come San Marco Group, ma che può essere generalizzato per altre aziende, dovrebbe necessitare di un proprio ufficio \textit{IT} deriva dal fatto che il bisogno di personalizzazione è talmente alto da rendere necessaria la presenza di un ufficio interno che segua lo sviluppo di un processo dall'inizio (esigenza) fino alla fine (messa in produzione) e che continui col mantenimento dello stesso.\\

%**************************************************************

%**************************************************************




















%\section{Casi d'uso}

%Per lo studio dei casi di utilizzo del prodotto sono stati creati dei diagrammi.
%I diagrammi dei casi d'uso (in inglese \emph{Use Case Diagram}) sono diagrammi di tipo \gls{uml} dedicati alla descrizione delle funzioni o servizi offerti da un sistema, così come sono percepiti e utilizzati dagli attori che interagiscono col sistema stesso.
%Essendo il progetto finalizzato alla creazione di un tool per l'automazione di un processo, le interazioni da parte dell'utilizzatore devono essere ovviamente ridotte allo stretto necessario. Per questo motivo i diagrammi d'uso risultano semplici e in numero ridotto.

%\begin{figure}[!h] 
%    \centering 
 %   \includegraphics[width=0.9\columnwidth]{usecase/scenario-principale} 
  %  \caption{Use Case - UC0: Scenario principale}
%\end{figure}

%\begin{usecase}{0}{Scenario principale}
%\usecaseactors{Sviluppatore applicativi}
%\usecasepre{Lo sviluppatore è entrato nel plug-in di simulazione all'interno dell'IDE}
%\usecasedesc{La finestra di simulazione mette a disposizione i comandi per configurare, registrare o eseguire un test}
%\usecasepost{Il sistema è pronto per permettere una nuova interazione}
%\label{uc:scenario-principale}
%\end{usecase}

%\section{Tracciamento dei requisiti}

%Da un'attenta analisi dei requisiti e degli use case effettuata sul progetto è stata stilata la tabella che traccia i requisiti in rapporto agli use case.\\
%Sono stati individuati diversi tipi di requisiti e si è quindi fatto utilizzo di un codice identificativo per distinguerli.\\
%Il codice dei requisiti è così strutturato R(F/Q/V)(N/D/O) dove:
%\begin{enumerate}
%	\item[R =] requisito
%    \item[F =] funzionale
%    \item[Q =] qualitativo
%    \item[V =] di vincolo
%    \item[N =] obbligatorio (necessario)
%    \item[D =] desiderabile
%    \item[Z =] opzionale
%\end{enumerate}
%Nelle tabelle \ref{tab:requisiti-funzionali}, \ref{tab:requisiti-qualitativi} e \ref{tab:requisiti-vincolo} sono riassunti i requisiti e il loro tracciamento con gli use case delineati in fase di analisi.

%\newpage

%\begin{table}%
%\caption{Tabella del tracciamento dei requisti funzionali}
%\label{tab:requisiti-funzionali}
%\begin{tabularx}{\textwidth}{lXl}
%\hline\hline
%\textbf{Requisito} & \textbf{Descrizione} & \textbf{Use Case}\\
%\hline
%RFN-1     & L'interfaccia permette di configurare il tipo di sonde del test & UC1 \\
%\hline
%\end{tabularx}
%\end{table}%

%\begin{table}%
%\caption{Tabella del tracciamento dei requisiti qualitativi}
%\label{tab:requisiti-qualitativi}
%\begin{tabularx}{\textwidth}{lXl}
%\hline\hline
%\textbf{Requisito} & \textbf{Descrizione} & \textbf{Use Case}\\
%\hline
%RQD-1    & Le prestazioni del simulatore hardware deve garantire la giusta esecuzione dei test e non la generazione di falsi negativi & - \\
%\hline
%\end{tabularx}
%\end{table}%

%\begin{table}%
%\caption{Tabella del tracciamento dei requisiti di vincolo}
%\label{tab:requisiti-vincolo}
%\begin{tabularx}{\textwidth}{lXl}
%\hline\hline
%\textbf{Requisito} & \textbf{Descrizione} & \textbf{Use Case}\\
%\hline
%RVO-1    & La libreria per l'esecuzione dei test automatici deve essere riutilizzabile & - \\
%\hline
%\end{tabularx}
%\end{table}%