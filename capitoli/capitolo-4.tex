% !TEX encoding = UTF-8
% !TEX TS-program = pdflatex
% !TEX root = ../tesi.tex


%**************************************************************
\chapter{Conclusioni}
\label{cap:conclusioni}
%**************************************************************

%\intro{Breve introduzione al capitolo}\\


\section{Raggiungimento degli obiettivi}

Gli obiettivi sono stati portati tutti a termine, nonostante sia mancata la possibilità di sviluppare meglio l'obiettivo desiderabile.
Infatti, nonostante il suo soddisfacimento non mi ritengo completamente appagato del risultato raggiunto. \\
Era mio desiderio approfondire maggiormente l'analisi dei dati con i soggetti interessati e capire in che modo possono essere usati per stimolare il cambiamento ed eliminare le inefficienze.\\
Questo è dovuto alla mancanza di tempo data dal prolungarsi dell'attività di verifica dei \textit{web service}, che hanno richiesto più tempo del previsto sia per la difficoltà personale che ho trovato nell'affrontare la tecnologia sconosciuta, che per alcuni problemi esterni all'azienda che hanno portato ad un rilascio posticipato del certificato digitale \textit{SSL} per il server web di Sage X3.


%**************************************************************


\section{Conoscenze acquisite}

\subsection{Conoscenze professionali}

La capacità di progettare e realizzare autonomamente un'applicazione web sfruttando
framework e tecnologie prima sconosciute, esperienza utile nell'analisi di un problema,
analisi e comprensione di un processo aziendale.

%**************************************************************


\subsection{Conoscenze personali}

Personalmente questa esperienza è stata utile per migliorare il mio modo di
rapportarmi, sviluppare una discreta capacità di problem solving considerando le varie
scelte fatte durante il progetto e la capacità di gestire più attività parallelamente.

%**************************************************************


\section{Valutazione personale}

Le conoscenze apprese durante il corso di studi sono state sufficienti per riuscire a
comprendere in autonomia l'utilizzo di framework o tecnologie non ancora esplorate. La parte
più difficile è stata rapportarsi con figure non tecniche e con esperienza lavorativa alle spalle, cercare di capire le necessità di attività molto lontane dall'ambiente scolastico/universitario finora affrontato. Penso che esperienze di questo tipo siano necessarie per comprendere il mondo del lavoro.

%**************************************************************

%**************************************************************




















%\section{Casi d'uso}

%Per lo studio dei casi di utilizzo del prodotto sono stati creati dei diagrammi.
%I diagrammi dei casi d'uso (in inglese \emph{Use Case Diagram}) sono diagrammi di tipo \gls{uml} dedicati alla descrizione delle funzioni o servizi offerti da un sistema, così come sono percepiti e utilizzati dagli attori che interagiscono col sistema stesso.
%Essendo il progetto finalizzato alla creazione di un tool per l'automazione di un processo, le interazioni da parte dell'utilizzatore devono essere ovviamente ridotte allo stretto necessario. Per questo motivo i diagrammi d'uso risultano semplici e in numero ridotto.

%\begin{figure}[!h] 
%    \centering 
 %   \includegraphics[width=0.9\columnwidth]{usecase/scenario-principale} 
  %  \caption{Use Case - UC0: Scenario principale}
%\end{figure}

%\begin{usecase}{0}{Scenario principale}
%\usecaseactors{Sviluppatore applicativi}
%\usecasepre{Lo sviluppatore è entrato nel plug-in di simulazione all'interno dell'IDE}
%\usecasedesc{La finestra di simulazione mette a disposizione i comandi per configurare, registrare o eseguire un test}
%\usecasepost{Il sistema è pronto per permettere una nuova interazione}
%\label{uc:scenario-principale}
%\end{usecase}

%\section{Tracciamento dei requisiti}

%Da un'attenta analisi dei requisiti e degli use case effettuata sul progetto è stata stilata la tabella che traccia i requisiti in rapporto agli use case.\\
%Sono stati individuati diversi tipi di requisiti e si è quindi fatto utilizzo di un codice identificativo per distinguerli.\\
%Il codice dei requisiti è così strutturato R(F/Q/V)(N/D/O) dove:
%\begin{enumerate}
%	\item[R =] requisito
%    \item[F =] funzionale
%    \item[Q =] qualitativo
%    \item[V =] di vincolo
%    \item[N =] obbligatorio (necessario)
%    \item[D =] desiderabile
%    \item[Z =] opzionale
%\end{enumerate}
%Nelle tabelle \ref{tab:requisiti-funzionali}, \ref{tab:requisiti-qualitativi} e \ref{tab:requisiti-vincolo} sono riassunti i requisiti e il loro tracciamento con gli use case delineati in fase di analisi.

%\newpage

%\begin{table}%
%\caption{Tabella del tracciamento dei requisti funzionali}
%\label{tab:requisiti-funzionali}
%\begin{tabularx}{\textwidth}{lXl}
%\hline\hline
%\textbf{Requisito} & \textbf{Descrizione} & \textbf{Use Case}\\
%\hline
%RFN-1     & L'interfaccia permette di configurare il tipo di sonde del test & UC1 \\
%\hline
%\end{tabularx}
%\end{table}%

%\begin{table}%
%\caption{Tabella del tracciamento dei requisiti qualitativi}
%\label{tab:requisiti-qualitativi}
%\begin{tabularx}{\textwidth}{lXl}
%\hline\hline
%\textbf{Requisito} & \textbf{Descrizione} & \textbf{Use Case}\\
%\hline
%RQD-1    & Le prestazioni del simulatore hardware deve garantire la giusta esecuzione dei test e non la generazione di falsi negativi & - \\
%\hline
%\end{tabularx}
%\end{table}%

%\begin{table}%
%\caption{Tabella del tracciamento dei requisiti di vincolo}
%\label{tab:requisiti-vincolo}
%\begin{tabularx}{\textwidth}{lXl}
%\hline\hline
%\textbf{Requisito} & \textbf{Descrizione} & \textbf{Use Case}\\
%\hline
%RVO-1    & La libreria per l'esecuzione dei test automatici deve essere riutilizzabile & - \\
%\hline
%\end{tabularx}
%\end{table}%