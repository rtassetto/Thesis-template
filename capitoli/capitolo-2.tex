% !TEX encoding = UTF-8
% !TEX TS-program = pdflatex
% !TEX root = ../tesi.tex

%**************************************************************
\chapter{Progetto di stage}
\label{cap:progetto-di-stage}
%**************************************************************


%**************************************************************
\section{Lo stage per San Marco Group}

Il concetto di stage formativo è molto importante per l’azienda San Marco Group.
In primo luogo viene considerata la sperimentazione di nuove tecnologie ed il loro confronto con quelle utilizzate quotidianamente. Per poterlo fare attualmente il personale presente dovrebbe smettere di svolgere le proprie attività, per questo motivo uno stagista universitario diventa una risorsa indispensabile, permettendo quindi all’azienda di sperimentare senza che vengano fermate le normali attività.
In secondo luogo, l’instaurazione del rapporto con l’Università di Padova è dato dalla possibilità di inserimento nell’azienda di risorse derivanti dal mondo universitario che, con una nuova prospettiva data dalle conoscenze apprese e messe in pratica nel corso di studio, possono portare idee creative e innovative, dalle quali può trarre beneficio anche il personale aziendale.


%**************************************************************
\section{Contesto attuale}

Attualmente l’azienda è nel vivo di un progetto di cambio gestionale. Per portare valore al passaggio, oltre che replicare nel nuovo gestionale le funzionalità già presenti in quello attuale, sono stati analizzati i processi che necessitano di un miglioramento.


%**************************************************************
\section{Proposta di stage}

Lo scopo dello stage consisteva nello sviluppo di un sistema di gestione di gare di offerte fornitori, dedicate a beni materiali e servizi, attualmente gestito tramite moduli cartacei e solo parzialmente all’interno dell’attuale gestionale.

%**************************************************************
\section{Analisi preventiva dei rischi}

Il rischio principale era dato dalla difficoltà di delineare il flusso corretto del processo a causa delle parecchie attività svolte extra sistema, che avrebbe portato alla creazione di una procedura incompleta o poco efficiente.


%**************************************************************
\section{Motivazione della scelta}

La scelta è ricaduta su questa proposta perché rispetto ad altre che ho ricevuto è parte di un progetto più grande come un cambio gestionale, che coinvolge molteplici processi aziendali e a mio parere ha un alto contenuto formativo. Ho avuto modo di rapportarmi direttamente con gli utenti in fase di analisi, per cercare di capire quali fossero i requisiti fondamentali da soddisfare. Inoltre, a conclusione dello stage avrò la possibilità di continuare a seguire le attività successive allo sviluppo del prodotto, come la formazione degli utenti, e vedere i risultati del mio lavoro nel lungo termine.